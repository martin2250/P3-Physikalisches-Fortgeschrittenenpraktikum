\chapter{NMR}
\todo{maybe mix up the first words a bit}
The experiment is conducted with a commercial NMR analyzer, the \texttt{Varian EM 360-A}, which uses an anlog chart recorder to store the spectrum.
The probe is placed in a constant, homogenous magnetic field which can be trimmed to improve accuracy.
Pressurized air is used to spin the probe, effectively making the field more homogenous. %really want to draw a parallel to a microwave stove
The perturbing frequency is controlled by the position of the chart recorder's head.
The sweep range and the center frequency can be selcted via rotary switches, the sweep duration can be selected to improve resolution or speed up the measurement.
To keep the experiment short, a quick scan is used to find peaks.
When recording the final spectrum, the sweep time is increased around the peaks and decreased where accuracy is not important.

\section{Calibrating the Magnetic Field}
Several small coils are placed around the permanent magnet, allowing fine adjustment of the magnetic field in the sample.
A vial of water is used as the sample and the field is trimmed as per the instructions provided.

\todo{did we record the width of the water line?}

\section{Acetic Acid (sample 1)}
The spectrum shows two distinctive peaks, corresponding to the hydrogen atoms in the methyl and carboxyl groups.
The peaks can be identified by their intensity, the methyl group has a higher intensity.
In this spectrum, the peak is twice as high as the carboxyl peak instead of the predicted ratio of three, this is possibly due to clipping when the recorder head hit the end stop.
The frequencies are \SI{2.18}{ppm} for the methyl group and \SI{11.5}{ppm} for the carboxyl group.
The high shift of the carboxyl group is due to the high electronegativity of oxygen, pulling the electrons close to the carboxyl group and shielding the single hydrogen nucleus from the external field.

\section{Booze (sample 2)}
In booze (also referred to as ethanol), there are three distinctive groups containing hydrogen atoms:

\center
{
	\begin{tabular}{cSSc}
		\toprule
		functional group&	{chemical shift $\delta$}&	{coupling constant $J$}\\
		\midrule
		hydroxy (\ce{OH})&	5.44& {singlet}\\
		alkyl (\ce{CH_2})&	3.81&	0.123&	4\\
		methyl (\ce{CH_3})&	1.22&	0.117&	3\\
		\bottomrule
	\end{tabular}
}

\section{Ethanol and Hydrochloric Acid}

\center
{
	\begin{tabular}{cSSc}
		\toprule
		functional group&	{chemical shift $\delta$}&	{coupling constant $J$}\\
		\midrule
		hydroxy (\ce{OH})&	suppressed\\
		alkyl (\ce{CH_2})&	3.84&	0.128&	4\\
		methyl (\ce{CH_3})&	1.28&	0.123&	3\\
		\ce{HCl}&	7.5& {singlet}\\
		\bottomrule
	\end{tabular}
}

\section{1- and 2-Propanol}

\center
{
	n-PrOH\\
	\begin{tabular}{cSSc}
		\toprule
		functional group&	{chemical shift $\delta$}&	{coupling constant $J$}\\
		\midrule
		methyl (\ce{CH_3})&	0.97&	0.116&	3\\
		alkyl A (\ce{CH_2})&	1.63&	0.116&	5\\
		alkyl B (\ce{CH_2})&	3.75&	\\
		hydroxy (\ce{OH})&	5.53&	\\
		\bottomrule
	\end{tabular}

	i-PrOH\\
	\begin{tabular}{cSSc}
		\toprule
		functional group&	{chemical shift $\delta$}&	{coupling constant $J$}\\
		\midrule
		2x methyl (\ce{CH_3})&	1.28&	0.103& 3\\
		alkyl (\ce{CH})&	4.19&	0.109&	5\\
		hydroxy (\ce{OH})&	5.28&	\\
		\bottomrule
	\end{tabular}
}

\section{}

\center
{
	\begin{tabular}{cSS}
		\toprule
		functional group&	{chemical shift $\delta$}&	{coupling constant $J$}\\
		\midrule

		\ce{HCl}&	7.5& {singlet}\\
		\bottomrule
	\end{tabular}
}
