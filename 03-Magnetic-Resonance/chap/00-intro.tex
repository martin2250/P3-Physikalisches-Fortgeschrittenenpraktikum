\chapter{Introduction}
Magnetic Resonance is used to find energies of possible quantum transitions in materials.
A sample in an oscillating field can absorb energy when the oscillating frequency coincides with the energy difference between two quantum states. %and some other rules are met (include?)
In both experiments, an external magnetic field is applied to lift the degeneracy of different magnetic states of particles that carry a magnetic dipole, creating the different energy states.

\section{Zeeman Effect}\label{sec:zeeman}
The \textsc{Zeeman} effect describes the splitting of a spectral line due to an external static magnetic field and is important in applications such as \textbf{ESR} and \textbf{NMR}, which will be discussed later on.

\subsection{Normal Zeeman Effect}
The effect that appears for spectral lines resulting from a transition between singlet states is called the \textbf{normal} effect.
For singlet states, the total angular momentum $\vec{J}$ is given solely by the angular momentum $\vec{L}$, which couples with the external magnetic field.

The change in energy is a function of this external field
\begin{equation*}
	\Delta E = -\mu_\text{z}\cdot B=m_l\mu_\text{B}B, \tag{$\vec{B}=B\hat{e}_\text{z}$}
\end{equation*}
where $\mu_\text{z}=-m_l\mu_\text{B}$ denotes the magnetic moment with the Bohr magneton $\mu_\text{B}=\frac{e\hbar}{2m_\text{e}}$.
Since there are $2l+1$ possible values for $m_l$, the energy levels split into $2l+1$ energy levels.
The selection rule $\Delta m_l=\pm 1$ reduces the number of possible lines.
For example, a transition between $l=2$ to $l=1$ states will result into nine lines.
However, as a consequence of the uniformity of the level splittings, only three different lines, each corresponding to the $\Delta m_l=-1,0,1$ transitions, are visible in the spectrum.

\subsection{Anomalous Zeeman Effect}
This effect appears for states with a nonzero total spin.
Consider an atom with the total spin $\vec{S}$ and orbital angular momentum $\vec{L}$.
We can then switch to a new basis $\ket{J, m_j}$ with $J=L+S$, whereas the total magnetic moment is given by
\begin{align*}
	\vec{\mu}_\text{tot} &= -\mu_\text{B}\left(g_\text{l}\frac{\vec{L}}{\hbar} + g_\text{s}\frac{\vec{S}}{\hbar}\right) \\
	&= -g_\text{J}\mu_\text{B}\frac{\vec{J}}{\hbar},
\end{align*}
where $g_\text{l}\approx 1$ and $g_\text{s}\approx 2$ are called the \textsc{Landé} factors.
By performing the vector addition of $\vec{L}$ and $\vec{S}$, $g_\text{J}$ takes the form
\begin{equation*}
	g_\text{j} = 1+\frac{J(J+1)+S(S+1)-L(L+1)}{2J(J+1)}.
\end{equation*}
Consider a doublet level transistion $\ce{^{2}P}_{1/2}\rightarrow \ce{^{2}S}_{1/2}$.
The selection rule $\Delta m_J =0, \pm 1$ gives 4 lines.

\section{ESR}\label{sec:esr}
Electron spin resonance or electron paramagnetic resonance is a spectroscopy meth%hehe, said meth
od for analyzing materials with unpaired electrons, such as organic radicals.
As discussed in \autoref{sec:zeeman}, if a sample is subjected to a static magnetic field, its energies split into Zeeman levels.
An unpaired electron can shift between energy levels by absorbing/emitting a circular polarized photon with frequency $\nu$, such that the resonance condition
\begin{equation}\label{eq:resonance}
	\Delta E = h\nu = g_\text{e}\mu_\text{B}B_0
\end{equation}
is satisfied. %unlike yo mama, she ugly.
The unpaired electron can now alter its energy state by either absorbing this photon or by stimulated emission.
Consider a two-level-system with energies $E_1$ and $E_2$ and $n=n_1+n_2$.
The population of the different energy levels obeys a Boltzmann distribution
\begin{equation*}
	\frac{n_2}{n_1}=\text{exp}\left(-\frac{E_2-E_1}{k_\text{B}T}\right).
\end{equation*}
Expanding the exponential function to the first order gives
\begin{equation*}
	\Delta n = \frac{n}{2}\cdot\frac{g_\text{e}\mu_\text{B}B_0}{k_\text{B}T}.
\end{equation*}
The lower-energy state is preferred minimally, so absorption will be observed most of the time.
The amount of absoprtion can be increased by lowering the temperature.

\subsection{Spectrum}
The form of the ESR spectrum can be influenced by many effects.
For example, an unpaired electron may interact with the magnetic moments of the nuclei, which is called \textbf{hyperfine-structure} and results into further splitting of energy levels.
Furthermore, internal \textbf{dipole-dipole interactions} which result into additional fluctuating magnetic fields, can influence the width of the ESR peak.
Fast fluctuations relative to the charactersitic frequency of dipole-dipole interaction give narrower peaks, whereas slow fluctuations result into wider peaks.
The main influence, however, surely is the external, \textbf{inhomogenous magnetic field}.
Although we make use of a Helmoltz coil, the homogenity of the magnetic field cannot be ensured at the edges of the relatively big sample.
The resonance condition is met at different frequencies throughout the sample, which effectively widens the peak drastically.
Furthermore, the \textbf{orientation of spins} as a consequence of an external magnetic field produces additional magnetic fields, which shift the resonance.
Moreover, if the distances between paramagnetic molecules is small enough for the wave functions to overlap, electrons may dislocate between molecules, which effectively decreases the effect of dipole-dipole peak widenings \cite{lab-inst}.

\section{NMR}
Nuclear Magnetic Resonance is used to analyze the different energy states that nuclei with a nonzero magnetic moment take in an external magnetic field.
NMR is used in medicine (MRI) or to detect isotopes of elements.
The resonance frequency is also affected by the electron hull of the atom (''nuclear shielding''), which makes it possible to analyze chemical structures with NMR.
Only nuclei that don't have an even number of protons and neutrons each can be analyzed, as protons and neutrons each form pairs that have opposing spin and thus have no net spin and magnetic moment.\\
In this experiment hydrogen atoms of different chemical compounds are analyzed.

\subsection{Resonance Frequency}

The energy of a hydrogen atom's proton's magnetic diopole $\mu_z = m \gamma \hbar$ in a magnetic field $B_0$ is
\begin{equation*}
	E = - \mu_z \cdot B_0 = m \gamma \hbar \cdot B_0,
\end{equation*}
where $m = \pm \tfrac{1}{2}$ denotes the magnetic quantum number and $\gamma$ the proton's gyromagnetic ratio.
The resulting resonance frequency is
\begin{equation*}
	f_0 = \frac{\Delta E}{h} = \frac{1}{2 \uppi} \gamma \cdot B_0,
\end{equation*}
which is \SI{60}{\mega\hertz} for the used field strength of \SI{1.41}{\tesla}.

\subsection{Nuclear Shielding}
A nucleus' environment can weaken the effective magnetic field that reaches the nucleus.
The resulting change in resonance frequency can be detected:
\begin{alignat*}{1}
	B_\text{eff} &= \left(1 - \sigma \right) \, B_0\\
	f_\text{r} &= \left(1 - \sigma \right) \, f_0
\end{alignat*}

\subsection{$\updelta$ Scale}
The exact resonance frequency is linearly dependent on the strength of the external magnetic field.
Calibrating this magnetic field to the required accuracies of $< \SI{1}{ppm}$ and accounting for temperature coefficients and drift is difficult at best.
Since only hydrogen atoms are analyzed and the expected resonance frequencies are very similar, frequencies are measured relative to the resonance frequency of the hydrogen atoms in a known chemical structure.

Tetramethylsilane proves to be a good reference compound, as it contains twelve hydrogens atoms which are chemically identical.

Frequencies are converted to the delta scale as
\begin{equation*}
	\delta = \frac{f - f_\text{TMS}}{f_\text{TMS}},
\end{equation*}
which is not dependent on $B_0$.
Values for delta are usually expressed in \si{ppm}.

\subsection{Spin-Spin Coupling}
The dipole moments of neighbouring nuclei can contribute to the external magnetic field.
Depending on the number of neighbouring atoms, this results in doublets, triplets or quaduplets instead of single peaks.
The frequency difference between peaks is independent of the external field, so it is specified directly.

For $n$ neighbouring, chemically equivalent nuclei $n + 1$ peaks are detected, with intensities proportional to ${{n + 1}\choose{m}}, m = 0 \dots n$.

Nuclei that are not chemically identical will result in more compliacated patterns.
