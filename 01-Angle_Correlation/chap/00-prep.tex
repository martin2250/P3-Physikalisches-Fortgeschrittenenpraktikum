\addchap{Prep}
\todo{no number or start at zero?}

This experiment validates \todo{actually validate and don't disprove} the anisotropy of the emission of gamma rays from excited nuclei states.
The angular distribution is dependent on the change of the magnetic quantum number between the inital and final states of the nucleus.
As the higher energy state has a spin number $j$ of at least one, there are multiple possible states which the excited nucleus can occupy.
At room temperature all possible states of the magnetic quantum number are populated equally, which results in an overall isotropic radiation pattern.

To isolate transitions with the same magnetic quantum number, a cascade of two transitions is observed, where the first transition leaves the nucleus in a defined state.
Naturally, the initial state also has multiple possible transitions which occur equally frequent, but the radiation pattern of the first transition is also anisotropic.

By selecting a nucleus with few enough states where there is a direction in which only one state emits radiation, it is possible use a first detector to detect this first gamma ray, confirming that the decaying nucleus is in the desired state, and observing the angular distribution with a second detector.

In this interpretation, the first detector also selects the quantization axis for the decay.
In reality, both detectors are interchangable.
Only the angle between the detectors is relevant for the analysis.
