\chapter{Data}

\section{Background}
\begin{table}
	\centering
	\caption[Background radiation]{\textbf{Background radiation}}
	\label{tab:background}
	\begin{tabular}{c|SSSS}
		\toprule
		background	&	{ch. 1 ev.}	&	{ch. 2 ev.}	&	{correlations}	&	{random coin.}	\\
		\midrule
			&	1181	&	1220	&	2.72	&	1.82	\\
		\bottomrule
	\end{tabular}
\end{table}

Background radiation data can be seen in \autoref{tab:background}, including distance corrections.
As data suggests, background radiation will not have a significant effect on the data.
In subsequent sections, background data will have been subtracted from measured quantities already.
\todo{Edit this. Blue book says different.}

\section{Measurement of $a_2$}
