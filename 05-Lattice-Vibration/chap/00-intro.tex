\chapter{Theory}
This experiment illustrates the different modes of vibration of a one-dimensional lattice by means of a mechanical model.
The objective of the experiment is to determine the dispersion relation $\omega(k)$ of this model and use it to conclude the mechanical properties of the setup.

\section{Group-/Phase velocity}
A wave can be characterized by its angular frequency $\omega$ and wave number $k$.
With these quantities, we can define its \textbf{phase velocity}
\begin{equation*}
	v_\text{ph}=\frac{\omega}{k},
\end{equation*}
the velocity at which same phases of a wave propagate in space.
In the same manner, a \textbf{group velocity}
\begin{equation*}
	v_\text{gr}=\frac{\d\omega}{\d k},
\end{equation*}
the speed at which a wave's modulation - hence information - travels through space is defined.
In a given medium, the frequency is some function $\omega(k)$, so in general, the group- and phase velocities depend on the medium and are equal only if the \textbf{refractive index} $n=\frac{c}{v_\text{ph}}$ is constant.

\section{The First Brillouin Zone}
The first Brillouin zone is the Wigner-Seitz-cell of the reciprocal lattice.
For a primitive cubic crystal system with lattice parameter $a$, the first Brillouin zone extends across the interval $k\in [-\frac{\pi}{a}, \frac{\pi}{a}]$ in reciprocal space.

Consider a linear chain of coupled atoms.
As a consequence of the discrete nature of this system, the number of physically distinguishable vibration modes is finite, where $\lambda_\text{min}=2a$ and $\lambda_\text{max}=2L$ ($L$ denotes the chain's length) are the minimum and maximum wavelengths.
Defining a \textbf{mode number} $n=\frac{2L}{\lambda_n}$, it follows that the maximal number of possible modes is
\begin{equation*}
	n_\text{max}=\frac{2L}{\lambda_\text{min}}=\frac{L}{a},
\end{equation*}
which is equal to the number of atoms in the chain.

In the same fashion, the maximum wave number is
\begin{equation*}
	k_\text{max}=\frac{2\pi}{\lambda_\text{min}}=\frac{\pi}{a},
\end{equation*}
which is equal to the boundaries of the first Brillouin zone of a primitive cubic lattice.
Therefore, the first Brillouin zone already contains every wave number $k$ that is relevant to the description of distiguishable vibration modes.

\section{Model}
A series of weights connected by springs is used as a model.
The weights ride on a linear air bearing and can move freely in one direction.
Several approximations are made in order to make this model possible:
\begin{itemize}
	\item The force between nodes is approximated by a harmonic potential,
	\item weights can only interact with immediate neighbors and
	\item any effects arising from quantum mechanics are disregarded.
\end{itemize}

\section{Identical Weights}
A chain of identical weights with mass $m$, linked by springs with stiffness $D$ and length $a$ is considered \todo{think of something better}.
The newtonian equation of motion for the displacement $x_j$ of the $j$-th element of the chain is
\begin{equation*}
	m \ddot x_j = D \cdot \left(x_{j - 1} + x_{j + 1} - 2 x_j \right).
\end{equation*}
The equation can be solved by the ansatz
\begin{equation*}
	x_j(t) = A \exp\left[\text{i} \left(k a j - \omega t\right)\right],
\end{equation*}
which yields the dispersion relation
\begin{equation}
	\omega(k) = \sqrt{\frac{4D}{m}}\abs{\sin\left(\frac{ka}{2}\right)}
\end{equation}

\section{Alternating Weights}
A chain of weights with alternating masses $m_1$ and $m_2$, linked by identical springs with stiffness $D$ and length $a$ is considered \todo{think of something better}.
The index $j$ denotes the $j$-th unit cell, which contains the weights $m_1$ with displacement $u_j$ and $m_2$ with displacement $v_j$.
The newtonian equations of motions are
\begin{gather*}
	m_1 \ddot u_j = D \cdot \left(v_{j - 1} + v_{j} - 2 u_j \right)\\
	m_2 \ddot v_j = D \cdot \left(u_{j} + u_{j + 1} - 2 v_j \right).
\end{gather*}
The equation can be solved by the ansatz
\begin{gather*}
	u_j(t) = U \exp\left[\text{i} \left(2 k a j - \omega t\right)\right]\\
	v_j(t) = V \exp\left[\text{i} \left(k a (2 j + 1) - \omega t\right)\right].
\end{gather*}
This yields the dispersion relation
\begin{equation}
	\omega_\pm^2(k) = \frac{D}{\mu} \pm \sqrt{\frac{D^2}{\mu^2} - \frac{4D^2}{m_1 m_2}\sin^2\left(k a\right)},
\end{equation}
with the reduced mass $\mu = \frac{m_1 m_2}{m_1 + m_2}$.

This dispersion relation reflects a property specific to crystals with different weight atoms, having distinct acoustic and photonic modes of vibraion.
In acoustic modes, all atoms share the same phase.
In optical modes, all atoms of each kind are out of phase by \SI{180}{\degree}.
Also there exists a photonic mode where the wave number is zero with finite angular frequency.
