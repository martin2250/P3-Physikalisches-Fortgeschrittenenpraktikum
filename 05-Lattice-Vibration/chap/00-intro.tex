\chapter{Theory}
This experiment illustrates the different modes of vibration of a one-dimensional lattice.
A series of weights connected by springs is used as a model.
The weights ride on a linear air bearing and can move freely in one direction.
The objective of the experiment is to determine the dispersion relation $\omega(k)$ of this model and use it to conclude the mechanical properties of the setup.

\section{Approximations}
Several approximations are made in order to make this model possible:
\begin{itemize}
	\item the force between nodes is approximated by a harmonic potential
	\item weights can only interact with immediate neighbours
\end{itemize}

\section{Identical Weights}
A chain of identical weights with mass $m$, linked by springs with stiffness $k$ and length $a$ is considered \todo{think of something better}.
The newtonian equation of motion for the displacement $x_j$ of the $j$-th element of the chain is
\begin{equation*}
	m \ddot x_j = k \cdot \left(x_{j - 1} + x_{j + 1} - 2 x_j \right).
\end{equation*}
The equation can be solved by the ansatz
\begin{equation*}
	x_j(t) = A \exp\left[\text{i} \left(k a j - \omega t\right)\right],
\end{equation*}
which yields the dispersion relation
\begin{equation}
	\omega(k) = \sqrt{\frac{4k}{m}}\abs{\sin\left(\frac{ka}{2}\right)}
\end{equation}

\section{Alternating Weights}
A chain of weights with alternating masses $m_1$ and $m_2$, linked by identical springs with stiffness $k$ and length $a$ is considered \todo{think of something better}.
The index $j$ denotes the $j$-th unit cell, which contains the weights $m_1$ with displacement $u_j$ and $m_2$ with displacement $v_j$.
The newtonian equations of motions are
\begin{gather*}
	m \ddot u_j = k \cdot \left(v_{j - 1} + v_{j} - 2 u_j \right)\\
	m \ddot v_j = k \cdot \left(u_{j} + u_{j + 1} - 2 v_j \right).
\end{gather*}
The equation can be solved by the ansatz
\begin{gather*}
	u_j(t) = U \exp\left[\text{i} \left(2 k a j - \omega t\right)\right]\\
	v_j(t) = V \exp\left[\text{i} \left(k a (2 j + 1) - \omega t\right)\right].
\end{gather*}
This yields the dispersion relation
\begin{equation}
	\omega_\pm^2(k) = \frac{k}{\mu} \pm \sqrt{\frac{k^2}{\mu^2} - \frac{4k}{m_1 m_2}\sin^2\left(k a\right)},
\end{equation}
with the reduced mass $\mu = \frac{m_1 m_2}{m_1 + m_2}$.
