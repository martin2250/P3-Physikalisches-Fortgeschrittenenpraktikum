\chapter{Data}
\section{Dispersion Relations}

\subsection{''Monoatomic'' Chain}
\begin{figure}
	\centering
	\includegraphics[width=0.7\textwidth]{./data/plots/dispersion_single_data.pdf}
	\caption[Dispersion Relation of Monoatomic Chain]{\textbf{Dispersion Relation of Monoatomic Chain} Edge of first Brillouin zone is marked in green.
	Dispersion function $\omega(k) = \sqrt{\frac{4D}{m}}\abs{\sin\left(\frac{ka}{2}\right)}$ with known mass $m$ and lattice parameter $a$ is fitted to the data.
	Found fit parameter: $D=\SI{27.05(9)}{\newton\per\meter}$. Goodness of fit: $\chi_\text{red}=\num{0.003}$.}
	\label{fig:dispersion_single}
\end{figure}

\subsection{''Biatomic'' Chain}
\begin{figure}
	\centering
	\includegraphics[width=0.7\textwidth]{./data/plots/dispersion_alternating_data.pdf}
	\caption[Dispersion Relation of Biatomic Chain]{\textbf{Dispersion Relation of Biatomic Chain} Edge of first Brillouin zone is marked in green.
	Dispersion functions $\omega_{-}$ and $\omega_{+}$ discussed in \autoref{sec:alt_theory} with known masses $m$, $M$ and lattice parameter $a$ are fitted to the data.
	Found fit parameters: $D_\text{opt}=\SI{24.6(3)}{\newton\per\meter}, D_\text{ac}=\SI{21.7(3)}{\newton\per\meter}$. Goodness of fit: $\chi_\text{opt, red}=\num{0.028}, \chi_\text{ac, red}=\num{0.007}$.}
	\label{fig:dispersion_single}
\end{figure}

\section{Amplitude Ratio of hetherogenous linear Chain}
The amplitude ratio $\frac{s_{o,m}}{s_{o,M}}$ is determined for each mode of the model.
For obvious reasons the amplitudes $s_{o,m}$ and $s_{o,M}$ cannot be measured at the same location in the chain.
Instead, the amplitudes $A_{5/6}$ of the fifth and sixth thingamajig are measured using the camera system.
The fifth thingamajig has mass $M$, the sixth has mass $m$.
To obtain the actual amplitude ratio of the envelope functions, the ratio $\frac{A_{6}}{A_{5}}$ is multiplied with the ratio of the local amplitudes of the envelopes at position five and six:
\begin{equation*}
	\frac{s_{o,m}}{s_{o,M}} = \frac{A_{6}}{A_{5}} \cdot \frac{\sin(\frac{\uppi n}{13} \cdot 5)}{\sin(\frac{\uppi n}{13} \cdot 6)}.
\end{equation*}


\todo{probably put this table somewhere else}

\begin{table}
	\centering
	\caption[Amplitude Ratios of hetherogenous linear Chain:]{\textbf{Amplitude Ratio of hetherogenous linear Chain:} The amplitudes $A_n$ and $B_n$ of two thingamajigs (5th and 6th thingamajig) are measured. The ratio is corrected for the different position in the chain to obtain the correct amplitude ratio between heavy and light thingamajigs.}
	\begin{tabular}{SSSSS}
		\toprule
		{$n$}&
		{$f$ (\si{\hertz})}&
		{$A_{5,n}$}&
		{$A_{6,n}$}&
		{Corrected Ratio}\\
		\midrule
		1&	0.240&	 143.1 \pm  0.1&	 143.1 \pm  0.1&	+0.99 \pm 0.00\\
		2&	0.477&	  90.6 \pm  0.1&	  90.6 \pm  0.1&	+0.99 \pm 0.00\\
		3&	0.705&	  96.9 \pm  0.3&	  96.9 \pm  0.5&	+0.92 \pm 0.00\\
		4&	0.919&	 166.9 \pm  1.6&	 166.9 \pm  2.8&	+0.86 \pm 0.03\\
		5&	1.109&	  39.1 \pm  0.2&	  39.1 \pm  0.3&	+0.64 \pm 0.00\\
		6&	1.248&	 126.1 \pm  1.5&	 126.1 \pm  0.6&	+0.30 \pm 0.00\\
		7&	1.659&	  15.3 \pm  0.5&	  15.3 \pm  0.5&	-5.16 \pm 0.19\\
		8&	1.755&	   8.5 \pm  0.9&	   8.5 \pm  0.7&	-2.32 \pm 0.29\\
		9&	1.866&	  49.1 \pm  0.5&	  49.1 \pm  0.5&	-1.79 \pm 0.02\\
		10&	1.963&	  22.1 \pm  0.5&	  22.1 \pm  0.5&	-1.61 \pm 0.03\\
		11&	2.035&	  33.7 \pm  0.4&	  33.7 \pm  0.6&	-1.50 \pm 0.04\\
		12&	2.078&	  56.0 \pm  0.3&	  56.0 \pm  0.4&	-1.60 \pm 0.01\\
		\bottomrule
	\end{tabular}
\end{table}
