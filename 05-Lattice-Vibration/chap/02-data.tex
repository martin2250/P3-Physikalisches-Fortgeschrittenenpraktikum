\chapter{Evaluation}

\section{Dispersion Relations}\label{sec:disp_rels}
\subsection{Statistical Errors}
Averarging over all four measurement series, where systematic errors on the frequency are deemed to be negligibly small, yields a statistical error of
\begin{equation*}
	\sigma_\text{avg,ij} = \frac{\sigma_{std,ij}}{\sqrt{4}}\quad (i,j)\in[1, 12]\otimes[1,2].
\end{equation*}

Propgating these errors into the average formation over both gliders yields
\begin{equation*}
	\sigma_{\omega,i} = \sqrt{\frac{1}{2}\sum_{j}\sigma_{avg,ij}}.
\end{equation*}

\subsection{''Monoatomic'' Chain}\label{subsec:disp_mono}
\begin{figure}
	\centering
	\includegraphics[width=0.7\textwidth]{./data/plots/dispersion_single_data.pdf}
	\caption[Dispersion Relation of Monoatomic Chain]{\textbf{Dispersion Relation of Monoatomic Chain} Edge of first Brillouin zone is marked in green.
	Dispersion function $\omega(k) = \sqrt{\frac{4D}{m}}\abs{\sin\left(\frac{ka}{2}\right)}$ with known mass $m$ and lattice parameter $a$ is fitted to the data.
	Found fit parameter: $D=\SI{24.95(40)}{\newton\per\meter}$. Goodness of fit: $\chi_\text{red}=\num{0.083}$.}
	\label{fig:dispersion_single}
\end{figure}
The measured dispersion relation for the monoatomic chain is shown in \autoref{fig:dispersion_single}.
Using the equation in \autoref{subsec:exp_mod_single} and the measured length $L=\SI{5.035(10)}{\meter}$ of the chain, the lattice parameter and edge of the first Brillouin zone are calculated as
\begin{align*}
	a &= \SI{0.420(1)}{\meter} \\
	k_\text{end} = \frac{\pi}{a} &= \SI{7.487(15)}{\per\meter}.
\end{align*}
Errors on $a$ and $k_\text{end}$ propagate linearly with the error on $L$.

\subsection{''Biatomic'' Chain}
\begin{figure}
	\centering
	\includegraphics[width=0.7\textwidth]{./data/plots/dispersion_alternating_data.pdf}
	\caption[Dispersion Relation of Biatomic Chain]{\textbf{Dispersion Relation of Biatomic Chain} Edge of first Brillouin zone is marked in green.
	Dispersion functions $\omega_{-}$ and $\omega_{+}$ discussed in \autoref{sec:alt_theory} with known masses $m$, $M$ and lattice parameter $a$ are fitted to the data.
	Found fit parameters: $D_\text{opt}=\SI{28.98(30)}{\newton\per\meter}, D_\text{ac}=\SI{29.32(102)}{\newton\per\meter}$. Goodness of fit: $\chi_\text{opt, red}=\num{0.022}, \chi_\text{ac, red}=\num{0.053}$.}
	\label{fig:dispersion_alternating}
\end{figure}
The biatomic dispersion relation is shown in \autoref{fig:dispersion_alternating}.
In the same fashion as in \autoref{subsec:disp_mono}, the lattice parameter and the edge of the first Brillouin zone are
\begin{align*}
	a &= \SI{0.839(2)}{\meter} \\
	k_\text{end} = \frac{\pi}{a} &= \SI{3.744(7)}{\per\meter}.
\end{align*}

\section{Speed of Sound}\label{sec:speed_of_sound}
Since the slope of the dispersion relations (discussed in \autoref{sec:disp_rels}) at the origin is the speed of sound, we may linearize the curves to get
\begin{align*}
	\frac{\Delta\omega}{\Delta k} = v_\text{mono} &= \SI{2.788(6)}{\meter\per\second} \\
	\frac{\Delta\omega}{\Delta k} = v_\text{bi} &= \SI{2.421(5)}{\meter\per\second},
\end{align*}
where $\Delta\omega = \omega_1$ and $\Delta k = \frac{\pi}{12a}$.

Errors on these values are calculated by Gaussian error propagation like
\begin{align*}
	\Delta v_\text{s} = \sqrt{\left(\frac{1}{\Delta k}\right)^2 \cdot \sigma_\omega^2 + \left(\frac{\Delta\omega\cdot n}{\pi}\right)^2 \cdot \sigma_\text{a}^2},
\end{align*}
where $\sigma_\omega$ and $\sigma_\text{a}$ denote the errors on frequency $\omega$, lattice parameter $a$ and $n=12$.

\section{Mass Ratio}
Using the results found in \autoref{sec:speed_of_sound}, the ratio of $m$ to $M$ may be calculated by the relation
\begin{equation*}
	\frac{M}{m} = 2\cdot\left(\frac{v_\text{mono}}{v_\text{bi}}\right)^2 -1 = \num{2.097(18)}.
\end{equation*}
This value is used to find
\begin{equation*}
	M = \SI{1.057(9)}{\kg}.
\end{equation*}
Errors on $v_\text{mono/bi}$ propagate into $M$ like
\begin{equation*}
	\Delta M = \sqrt{ \left(\frac{4mv_\text{mono}}{v_\text{bi}^2}\right)^2 \cdot (\Delta v_\text{mono})^2 + \left(\frac{4mv_\text{mono}^2}{v_\text{bi}^3}\right)^2 \cdot (\Delta v_\text{bi})^2 }.
\end{equation*}

\section{Stiffness of Springs}
The stiffness of springs used in this model experiment is determined using two different approaches.

\subsection{Using Fit}
\autoref{fig:dispersion_single} and \autoref{fig:dispersion_alternating} show the recorded dispersion relations for the monoatomic and biatomic chain, respectively.
Instead of considering two pairs of values, the models discussed in sections \ref{sec:single_theory} and \ref{sec:alt_theory} are fitted to the data, yielding the stiffness as the only fit parameter.
Evaluating data in this manner is expected to give more accurate results, since all measurements are considered.
We find
\begin{align*}
	D_\text{mono} &= \SI{24.95(40)}{\newton\meter^{-1}} \\
	\left\{ \begin{array}{@{}ll@{}}
		D_\text{bi, opt} = \SI{28.98(30)}{\newton\meter^{-1}} \\
		D_\text{bi, ac} = \SI{29.32(102)}{\newton\meter^{-1}}
	\end{array}\right\} \Rightarrow \bar{D}_\text{bi} &= \left(\num{29.15}\pm \num{0.12}\ \text{(std)}\pm \num{0.53}\ \text{(stat)}\right)\si{\newton\meter^{-1}}.
\end{align*}
The stiffness found for the biatomic chain deviates significantly from the value found for the monoatomic chain (\num{16.8}\%). \todo{Should I discuss this issue?}

\subsection{Using $v_\text{s}$}
Rearranging the expressions in \ref{eq:sound_single} and \ref{eq:sound_alternating}, it is possible to calculate the stiffness as
\begin{alignat*}{3}
	D_\text{mono} &=\qquad \frac{m\cdot v_\text{mono}^2}{a^2} &&=  \SI{22.25(13)}{\newton\meter^{-1}} \\
	D_\text{bi} &= \frac{2(m+M)\cdot v_\text{bi}^2}{a^2} &&= \SI{25.99(21)}{\newton\meter^{-1}}. \\
\end{alignat*}
Errors on $v_\text{mono/bi}$, $M$ and $a$ propagate into $D_\text{mono/bi}$ like
\begin{align*}
	\Delta D_\text{mono} &= \sqrt{ \left(\frac{2mv_\text{mono}}{a^2}\right)^2 \cdot(\Delta v_\text{mono})^2 + \left(\frac{2mv_\text{mono}^2}{a^3}\right)^2 \cdot (\Delta a)^2 } \\
	\Delta D_\text{bi} &= \sqrt{ \left(\frac{2v_\text{bi}^2}{a^2}\right)^2 \cdot (\Delta M)^2
	+ \left(\frac{4(m+M)v_\text{bi}}{a^2}\right)^2 \cdot (\Delta v_\text{bi})^2
	+ \left(\frac{4(m+M)v_\text{bi}^2}{a^3}\right)^2 \cdot (\Delta a)^2 }.
\end{align*}

\section{Amplitude Ratio of Biatomic Linear Chain}
The amplitude ratio $\frac{s_{o,m}}{s_{o,M}}$ is determined for each mode of the model.
For obvious reasons the amplitudes $s_{o,m}$ and $s_{o,M}$ cannot be measured at the same location in the chain.
Instead, the amplitudes $A_{5/6}$ of the fifth and sixth slider are measured using the camera system.
The fifth slider has mass $M$, the sixth has mass $m$.
To obtain the actual amplitude ratio of the envelope functions, the ratio $\frac{A_{6}}{A_{5}}$ is multiplied with the ratio of the local amplitudes of the envelopes at position five and six:
\begin{equation*}
	\frac{s_{o,m}}{s_{o,M}} = \frac{A_{6}}{A_{5}} \cdot \frac{\sin(\frac{\uppi n}{13} \cdot 5)}{\sin(\frac{\uppi n}{13} \cdot 6)}.
\end{equation*}

To observe single modes, the previously fixed left end of the chain is driven by a stepper motor that rotates at the frequencies found in the previous experiments.
After starting the motor, the system is left for several minutes before taking the measurements, to let any oscillations of the other modes die down.
In total, 20 successive measurements are taken for each mode, the first of which is discarded as the \texttt{LabView} script produces a wrong reading immediately after starting the mesasurement.

The ratio $\frac{A_{6}}{A_{5}}$ is calculated for all nineteen pairs of amplitudes in each measurement series individually.
The error on the final amplitude ratio is simply the standard deviation of the individual measurements divided by the square root of the number of measurements.

\begin{table}
	\centering
	\caption[Amplitude Ratios of Biatomic Linear Chain:]{\textbf{Amplitude Ratio of Biatomic Linear Chain:} The amplitudes $A_{5,n}$ and $A_{6,n}$ of two sliders (5th and 6th slider) are measured. The ratio is corrected for the different position in the chain and the missing sign to obtain the correct amplitude ratio between heavy and light sliders.}
	\begin{tabular}{SSSSS[retain-explicit-plus]}
		\toprule
		{Mode $n$}&
		{$f$ (\si{\hertz})}&
		{$A_{5,n}$}&
		{$A_{6,n}$}&
		{Corrected Ratio}\\
		\midrule
		  1&	0.240&	 143.1 \pm  0.0&	 150.7 \pm  0.0&	+0.99 \pm 0.00\\
		  2&	0.477&	  90.6 \pm  0.0&	  32.4 \pm  0.0&	+0.99 \pm 0.00\\
		  3&	0.705&	  96.9 \pm  0.1&	 180.3 \pm  0.1&	+0.92 \pm 0.00\\
		  4&	0.919&	 166.9 \pm  0.5&	  67.2 \pm  0.8&	+0.86 \pm 0.01\\
		  5&	1.109&	  39.1 \pm  0.0&	  86.1 \pm  0.1&	+0.64 \pm 0.00\\
		  6&	1.248&	 126.1 \pm  0.4&	  30.2 \pm  0.2&	+0.30 \pm 0.00\\
		  7&	1.659&	  15.3 \pm  0.2&	  63.3 \pm  0.1&	-5.16 \pm 0.06\\
		  8&	1.755&	   8.5 \pm  0.3&	  67.2 \pm  0.2&	-2.32 \pm 0.08\\
		  9&	1.866&	  49.1 \pm  0.1&	  41.3 \pm  0.1&	-1.79 \pm 0.01\\
		 10&	1.963&	  22.1 \pm  0.1&	  71.7 \pm  0.1&	-1.61 \pm 0.01\\
		 11&	2.035&	  33.7 \pm  0.1&	  18.3 \pm  0.2&	-1.50 \pm 0.01\\
		 12&	2.078&	  56.0 \pm  0.1&	  95.2 \pm  0.1&	-1.60 \pm 0.00\\
		\bottomrule
	\end{tabular}
\end{table}

\begin{figure}
	\centering
	\includegraphics[width=0.7\textwidth]{./data/plots/amplitude-ratio.pdf}
	\caption[Amplitude Ratio of biatomic Chain]{\textbf{Amplitude Ratio of biatomic Chain}, theoretical curves are fitted to the data, found parameter $\frac{M}{m} = \num{1.39(2)}$ (acoustic) and $\frac{M}{m} = \num{1.46(2)}$ (optic). The last value of the optical modes at the edge of the first Brillouin zone is not used for the fit as the model predicts a singularity there.}
	\label{fig:amplitude-ratio}
\end{figure}

\section{Error Sources}
This experiment hosts many sources for errors of unknown magnitude that we are not able to account for.
The most prominent of which is that only the mean mass of all sliders is given, so it is not possible to make assumptions regarding how much the individual sliders vary from their mean value.
Also, in the calculation the springs are assumed to have no mass.
To make this assumption valid, the mass of the springs should be meausred to be negligibly small compared to the mass of the sliders.
