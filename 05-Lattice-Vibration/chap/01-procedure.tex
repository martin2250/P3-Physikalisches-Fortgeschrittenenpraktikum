% !TEX root = ../05-Lattice-Vibration.tex
\chapter{Procedure}
\section{Model}
The model consists of twelve thingimajigs and thirteen springs.
The first spring is connected to a stepper motor that can creates reciprocal harmonic motion at a selectable frequency and presents a fixed end to the chain.
The last spring is fixed to the end of the rail.

The thingimajigs consist of a piece of angle aluminum and have a threaded hole to mount additional weights.
Each thingimajig has a stripe of retroreflective material glued to it's side.
They ride on a linear air bearing which is a square tube sealed on both ends oriented on one of it's edges.
A turbine sends an air stream into the tube.
The tube's two sides facing upwards have a series of small holes to create an air cushion between the tube and the thingimajigs.
To dampen the oscillation after an experiment, the turbine's air inlet is blocked briefly.

\section{Data Aquisition}
A camera with a horizontal linear CCD is used to capture the position the retroreflective stripes on the fifth and sixth thingimajig.
Prior to the experiment the camera is aligned so the retroreflectors are in the plane of the CCD.
A \texttt{LabView} program is used to read the two positions from the camera.
The program is then used to apply a fast fourier transform to the positions to find the frequencies of the individual modes and later to measure the amplitudes of single modes.
