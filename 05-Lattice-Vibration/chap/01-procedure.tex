% !TEX root = ../05-Lattice-Vibration.tex
\chapter{Procedure}
\section{Model}
A model of twelve sliders and thirteen springs is used.
The first spring is connected to a stepper motor that can creates reciprocal harmonic motion at a selectable frequency and presents a fixed end to the chain.
The last spring is fixed to the end of the rail.

The sliders consist of a piece of angle aluminium and have a threaded hole to mount additional weights.
Each slider has a stripe of retroreflective material glued to its side.
They ride on a linear air bearing, which is a square tube sealed on both ends oriented on one of its edges.
A turbine sends an air stream into the tube.
The tube's two sides facing upwards have a series of small holes to create an air cushion between the tube and the sliders.
To dampen the oscillation after an experiment, the turbine's air inlet is blocked briefly.

\section{Data Acquisition}\label{sec:data_acq}
A camera with a horizontal linear CCD is used to capture the position the retroreflective stripes on the fifth and sixth slider.
Prior to the experiment, the camera is aligned so the retroreflectors are in the plane of the CCD.
A \texttt{LabView} program is used to read the two positions from the camera.
The program is then used to apply a fast fourier transform to the positions to find the frequencies of the individual modes and later to measure the amplitudes of single modes.
