\chapter{Procedure}

\secation{Spectroscope}
The experiment ist performed using a russian spectroscope.
Incoming light is directed via mirrors to a diffraction grating.
More mirrors are used to direct the light from a certain angle of the diffraction grating onto a photomultiplier tube.
A synchronous motor with selectable gearing is used to move the mirrors and sweep the observed angle.
The PMT's output is digitized and plotted on a PC.
The X-axis input is simply the time since the spectrogram (and therefore the motor) was started, which is scaled according to the selected gearing.

Special attention must be paid to the system's backlash, every time the motor changes direction there is a deadband of $\approx\SI{300}{\angstrom}$ which has to be taken into account when resetting the system for a new measurement.

\section{Helium Light Source}
A thin tube of low pressure helium is used as the light source.
Two electrodes are used to apply a high voltage to the gas.
The tube is placed in the gap of a large electromagnet yoke.
A lab power supply with constant current capabilites is used to generate a variable constant current through the electromagnet.
