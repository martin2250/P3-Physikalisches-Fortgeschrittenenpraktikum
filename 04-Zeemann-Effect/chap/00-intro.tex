\chapter{Theory}
\section{Zeeman Effect}\label{sec:zeeman}
The \textsc{Zeeman} effect describes the splitting of a spectral line due to an external static magnetic field.

\subsection{Normal Zeeman Effect}
The effect that appears for spectral lines resulting from a transition between singlet states is called the \textbf{normal} effect.
For singlet states, the total angular momentum $\vec{J}$ is given solely by the angular momentum $\vec{L}$, which couples with the external magnetic field.

The change in energy is a function of this external field
\begin{equation*}
	\Delta E = -\mu_\text{z}\cdot B=m_l\mu_\text{B}B, \tag{$\vec{B}=B\hat{e}_\text{z}$}
\end{equation*}
where $\mu_\text{z}=-m_l\mu_\text{B}$ denotes the magnetic moment with the Bohr magneton $\mu_\text{B}=\frac{e\hbar}{2m_\text{e}}$.
Since there are $2l+1$ possible values for $m_l$, the energy levels split into $2l+1$ energy levels.
The selection rule $\Delta m_l=\pm 1$ reduces the number of possible lines.
For example, there are nine transitions from $l=2$ to $l=1$ states.
However, as a consequence of the uniformity of the level splittings, only three different lines, corresponding to the $\Delta m_l=-1,0,1$ transitions, are visible in the spectrum.

\subsection{Anomalous Zeeman Effect}
This effect appears for states with a nonzero total spin.
Consider an atom with the total spin $\vec{S}$ and orbital angular momentum $\vec{L}$.
We can then switch to a new basis $\ket{J, m_j}$ with $J=L+S$, where the total magnetic moment is given by
\begin{align*}
	\vec{\mu}_\text{tot} &= -\mu_\text{B}\left(g_\text{l}\frac{\vec{L}}{\hbar} + g_\text{s}\frac{\vec{S}}{\hbar}\right) \\
	&= -g_\text{J}\mu_\text{B}\frac{\vec{J}}{\hbar},
\end{align*}
where $g_\text{l} = 1$ and $g_\text{s}\approx 2$ are called the \textsc{Landé} factors.
By performing the vector addition of $\vec{L}$ and $\vec{S}$, $g_\text{J}$ takes the form
\begin{equation*}
	g_\text{j} = 1+\frac{J(J+1)+S(S+1)-L(L+1)}{2J(J+1)}.
\end{equation*}
Consider a doublet level transistion $\ce{^{2}P}_{1/2}\rightarrow \ce{^{2}S}_{1/2}$.
The selection rule $\Delta m_J =0, \pm 1$ gives 4 lines.

\section{Paschen-Back Effect}
This strong-field limit of the Zeeman effect appears when the external magnetic field perturbation significantly exceeds spin-orbit interaction.
The magnetic field effect dominates over the coupling between orbital $\left(\vec{L}\right)$ and spin $\left(\vec{S}\right)$ angular momenta, thus, it is safe to assume that
\begin{equation*}
	\comm{H_0}{S} = 0,
\end{equation*}
where $H_0$ denotes the unperturbed Hamiltonian of the system.
Therefore, the energy eigenvalues for a state $\ket{\psi}$ can simply be calculated as
\begin{align*}
	E_\text{PB} &= \ev{H_0+\frac{\mu_\text{B}B_\text{z}}{\hbar}\left(g_\text{L}L+g_\text{S}S\right)}{\psi} \\
	&= E_0 + \mu_\text{B}B_\text{z}\left(g_\text{L}m_\text{L}+g_\text{S}m_\text{S}\right).
\end{align*}

\subsection{Polarizations}

\section{Electron Mass-to-Charge Ratio}

\section{Term Scheme of Helium}
Helium contains 2 electrons, whose spins can be aligned parallel (orthohelium) or anti-parallel (parahelium), where orthohelium is a triplet state and parahelium a singlet state.
The term diagram of helium consists of these two states' term shemes, which do not combine because of the selection rule $\Delta S = 0$.

The singlet state has no fine structure, since its total spin $S$ is equal to 0.
For the triplet state, however, the total spin is equal to 1.
Therefore, the state splits into three different energy levels and features a fine structure as a consequence of a non-zero total spin.

Orthohelium's ground state is a $2S$ state, since the Pauli exclusion principle forbids the electrons from occupying the $1S$-state as they have the same spin quantum number.
Parahelium, on the other hand, can occupy the $1S$-state.

The selection rules are
\begin{align*}
	\Delta J &= 0, \pm 1 \\
	\Delta L &= 0, \pm 1 \\
	\Delta S &= 0.
\end{align*}
