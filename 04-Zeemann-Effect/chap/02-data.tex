\chapter{Data}
\section{Overview Spectrum}
\begin{figure}[!h]
	\centering
	\includegraphics[width=.7\textwidth]{./data/external/300nm_800nm.pdf}
	\caption[Overview Spectrum: Spectrogram]{\textbf{Overview Spectrum: Spectrogram} Peaks are highlighted in plum.}
	\label{fig:overview300}
\end{figure}
\begin{figure}[tp]
	\centering
	\includegraphics[width=.6\textwidth]{./data/external/500nm_800nm.pdf}
	\caption[\SIrange{500}{800}{\nm}-Spectrum: Spectrogram]{\textbf{\SIrange{500}{800}{\nm}-Spectrum: Spectrogram} Peaks are highlighted in plum.}
	\label{fig:overview500}
\end{figure}
\begin{table}
	\centering
	\caption[Overview Spectrum: Peaks]{\textbf{Overview Spectrum: Peaks}}
	\label{tab:peaks300}
	\begin{tabular}{SS|c}
		\toprule
		{$\lambda$ (\si{\angstrom})}	&	{$\lambda_\text{lit}$ (\si{\angstrom})}	&	transition	\\
		\midrule
		3909.9	&	3888.65	&	$3p\rightarrow 2s$ \\
		4490.05	&	4471.48	&	$4d\rightarrow 2p$ \\
		4946.41	&	4921.93	&	$4d\rightarrow 2p$ \\
		5039.27	&	5047.74	&	$4S\rightarrow 2P$ \\
		5891.58	&	5875.62	&	$3d\rightarrow 2p$ \\
		6693.49	&	6678.15	&	$3D\rightarrow 2P$ \\
		7079.65	&	7065.71	&	$3s\rightarrow 2p$ \\
		7801.35	&	{$2\times3888.65$}	&	$3p\rightarrow 2s$ (2nd order) \\
		\bottomrule
	\end{tabular}
\end{table}
\begin{table}
	\centering
	\caption[\SIrange{500}{800}{\nm}-Spectrum: Peaks]{\textbf{\SIrange{500}{800}{\nm}-Spectrum: Peaks} With applied red filter, only first-order peaks are observed.}
	\label{tab:peaks500}
	\begin{tabular}{SS|c}
		\toprule
		{$\lambda$ (\si{\angstrom})}	&	{$\lambda_\text{lit}$ (\si{\angstrom})}	&	transition	\\
		\midrule
		5888.48	&	5875.62	&	$3d\rightarrow 2p$ \\
		5909.83	&	5875.64	&	$3d\rightarrow 2p$ \\
		6690.02	&	6678.15	&	$3D\rightarrow 2P$ \\
		7076.18	&	7065.71	&	$3s\rightarrow 2p$ \\
		7292.1	&	7281.35	&	$3S\rightarrow 2P$ \\
		\bottomrule
	\end{tabular}
\end{table}
To gather general information about the considered helium spectrum, an overview spectrum in the wavelength range $\lambda\in [\SI{300}{\nm}, \SI{800}{\nm}]$ is recorded.\todo{Doubling of 'spectrum'.}
The gap opening is \SI{100}{\um}, while the grating rotates at a speed of \SI{200}{\angstrom\per\minute}.
\autoref{fig:overview300} depicts the spectrogram, peaks are listed in \autoref{tab:peaks300} with their corresponding transitions \cite{hetrans}.
Only peaks of first and second order are examined.

\section{\SIrange{500}{800}{\nm}-Spectrum}
A red filter is placed between the gap opening of the spectrometer and the helium light source which serves the purpose of filtering out higher order peaks.
\autoref{fig:overview500} depicts the spectrogram, peaks are listed in \autoref{tab:peaks500} with their corresponding transitions \cite{hetrans}.
It is easy to see, that the list in \autoref{tab:peaks500} does not feature the $2\times\SI{3888.65}{\nm} = \SI{7801.35}{\nm}$ transition, which is observed in the overview spectrum without red filter.


\section{Mass-to-Charge Ratio}
A spectrogram is recorded for each particular field strength within a range of \SIrange{0.4}{1}{\tesla} in \SI{0.1}{\tesla} steps.
Measurements are listed in \autoref{tab:meas}.
Individual peaks are not discernible for $B = \SI{0.4}{\tesla}$, so this measurement is not listed.

\begin{table}[tbp]
	\centering
	\caption[Measured Zeeman levels ($B = \num{0.5} \dots \SI{1}{T}$)]{\textbf{Measured Zeeman levels ($B = \num{0.5} \dots \SI{1}{T}$)}, description}
	\label{tab:meas}
	\begin{tabular}{SSS}
		\toprule
		{$B$ (\si{\tesla})}& {$\lambda_+$ (\si{\angstrom})}& {$\lambda_-$ (\si{\angstrom})}\\
		\midrule
		0.5&	6676.6&	6676.4\\
		0.6&	6676.8&	6676.5\\
		0.7&	6676.9&	6676.6\\
		0.8&	6676.8&	6676.4\\
		0.9&	6676.9&	6676.4\\
		1.0&	6676.9&	6676.5\\
		\bottomrule
	\end{tabular}
\end{table}

The mass-to-charge ratio is calculated from the energy difference of the Zeeman levels via the Bohr magneton.
Individual peaks are usually drowned out by the center ($\Delta m = 0$) peak, so a $\upsigma$-filter is used to block the center peak.
The positions of the outer ($\Delta m = \pm 1$) peaks are used to obtain $\frac{e}{m_\text{e}}$:
\begin{alignat}{2}
	& &E &= E_0 \pm \mu_\text{B} g_\text{L} B\\
	&\Leftrightarrow \quad &\underbrace{\frac{c h}{\lambda_+} - \frac{c h}{\lambda_-}}_{\Delta E} &=
	2 \cdot \left(\frac{e \hbar}{2 m_\text{e}} \cdot B \right)\\
	&\Leftrightarrow &\frac{e}{m_\text{e}} &= \frac{2 \uppi c}{B} \cdot \left(\frac{1}{\lambda_+} - \frac{1}{\lambda_-}\right)\\
	&\Leftrightarrow &\frac{e}{m_\text{e}} &= \frac{2 \uppi c}{B} \cdot \left(\frac{\Delta \lambda}{\lambda^2}\right)
\end{alignat}
with $g_\text{L} = 1$.

\todo{Too much 'the' in the following paragraphs, reword.}
The only quantities for which an error can be specified are $\lambda_\pm$ and $B$.
The error on $\lambda_\pm$ is further split into the error on the absolute ($\lambda^2$) and relative ($\Delta \lambda$) value.

The error on the mean wavelength $\lambda$ is mostly dependent on the calibration of the spectroscope.
There are no error bounds provided, so the value for $\lambda$ is assumed to be exact. \todo{Rather than assuming the wavelength to be exact, it would be better to assume that 'the uncertainty is negligably small.'}
Comparing the average mean wavelength from the data $\lambda = \SI{6676.6}{\angstrom}$ with the literature value $\lambda_\text{lit} = \SI{6678.1517}{\angstrom}$ justifies this assumpion.

The relative wavelength $\Delta \lambda$ is mostly influenced by the error between the mismatch of the speed of the diffraction grating's rotation and the computer's internal clock.
The computer's clock can safely be assumed to be exact.
The speed of the synchronous motor is affected by the frequency of mains voltage.
Mains frequency is stable to within $\SI{\pm 100}{\milli\hertz}$ short term.
This statistic error propagates linearly, resulting in a $\SI{0.2}{\percent}$ error on $\Delta\lambda$.

The magnetic field is dependent on the accuracy of the provided $I-B$ chart, the shunt resistor and the voltmeter.
As an analog instrument is used to read the current, a statistic error of $\SI{\pm 0.05}{\tesla}$ is assumed.
It is also possible that the current through the magnet changes with time due to resistive heating.
As the power supply was operated in constant current mode, this effect can be neglected.
This was also confirmed by reading the current again at the end of the measurement.

The errors on mains frequency and the magnetic field are uncorrelated, so gaussian error propagation is used. \todo{Do explicit calculation. They want that. :(}
